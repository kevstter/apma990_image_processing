\chapter{Conclusion}
\label{ch:con}

From the beginning to the end, we sought to provide an overview of image segmentation, building from a mathematical modelling perspective which then gave way to numerics, discretization, implementation and practical performance. 

Starting from the review of two influential image segmentation models, GAC and ACWE, we examined key modelling principles which were then unified into a convex segmentation model. We investigated the use of two numerical methods to generate solutions with the first involving the Euler-Lagrange equations and the second the Split Bregman method. Driving the Euler-Lagrange equation to steady state proved to be a powerful technique that could be applied to all three models considered. The Split Bregman, a convex optimization algorithm for $\ell_1$-minimization problems, outperformed the previous method by a wide margin where it was applicable, namely with the GCS model. That, however, was not the end of the story as our examples showed there is a range of parameters where the Split Bregman is blazingly fast, but there is also a range for which the Split Bregman performs sluggishly.

Looking forward, we have already mentioned the need to accelerate the solution procedure when $\lambda$ is small. Would a different optimization algorithm be suitable or is there some way to incorporate a similar idea that was used to accelerate the evolution of the GAC model? It was also shown that a similar convexification procedure could be applied to the piecewise smooth Mumford-Shah model, and in \cite{bresson2007fast} the authors proposed solution via a dual gradient projection method. Exploring the use of the Split Bregman there may be worthwhile.
