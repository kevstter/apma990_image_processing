\documentclass[undefended]{sfuthesis}

\title{A Unified Convex Segmentation Model and Fast Implementation with the Split Bregman Method}
\author{Kevin Chow}
\discipline{Mathematics}
\department{Department of Mathematics}
\faculty{Faculty of Science}
\copyrightyear{2020}
\semester{Spring 2020}
\date{\today}

%\keywords{Do I need an abstract? This can come last.}

%   PACKAGES %%%%%%%%%%%%%%%%%%%%%%%%%%%%%%%%%%%%%%%%%%%%%%%%%%%%%%%%%%%%%%%%%%
%
%   Add any packages you need for your thesis here.
%   You don't need to call the following packages, which are already called in
%   the sfuthesis class file:
%
%   - appendix
%   - etoolbox
%   - fontenc
%   - geometry
%   - lmodern
%   - nowidow
%   - setspace
%   - tocloft
%
%   If you call one of the above packages (or one of their dependencies) with
%   options, you may get a "Option clash" LaTeX error. If you get this error,
%   you can fix it by removing your copy of \usepackage and passing the options
%   you need by adding
%
%       \PassOptionsToPackage{<options>}{<package>}
%
%   before \documentclass{sfuthesis}.
%
%   The following packages are a few suggestions you might find useful.
%
%   (1) amsmath and amssymb are essential if you have math in your thesis;
%       they provide useful commands like ``blackboard bold'' symbols and
%       environments for aligning equations.
%   (2) amsthm includes allows you to easily change the style and numbering of
%       theorems. It also provides an environment for proofs.
%   (3) graphicx allows you to add images with \includegraphics{filename}.
%   (4) hyperref turns your citations and cross-references into clickable
%       links, and adds metadata to the compiled PDF.
%   (5) pdfpages lets you import pages of external PDFs using the command
%       \includepdf{filename}. You will need to do this if your research
%       requires an Ethics Statement.
%

\usepackage{amsmath,amssymb,amsthm}                 % (1)
\usepackage{graphicx}\graphicspath{{figs/}}                       % (3)
\usepackage[pdfborder={0 0 0}]{hyperref}        % (4)
% \usepackage{pdfpages}                         % (5)
% ...
% ...
% ...
% ... add your own packages here!
\usepackage{subcaption}
\usepackage{cleveref}
\usepackage{mathtools}

\usepackage{booktabs}
\usepackage{siunitx}



%   OTHER CUSTOMIZATIONS %%%%%%%%%%%%%%%%%%%%%%%%%%%%%%%%%%%%%%%%%%%%%%%%%%%%%%
%
%   Add any packages you need for your thesis here. We've started you off with
%   a few suggestions.
%
%   (1) Use a single word space between sentences. If you disable this, you
%       will have to manually control spacing around abbreviations.
%   (2) Correct the capitalization of "Chapter" and "Section" if you use the
%       \autoref macro from the `hyperref` package.
%   (3) The LaTeX thesis template defaults to one-and-a-half line spacing. If
%       your supervisor prefers double-spacing, you can redefine the
%       \defaultspacing command.
%

\newtheorem{thm}{Theorem}

\theoremstyle{definition}
\newtheorem{eg}{Example}

\frenchspacing                                    % (1)
\renewcommand*{\chapterautorefname}{Chapter}      % (2)
\renewcommand*{\sectionautorefname}{Section}      % (2)
\renewcommand*{\subsectionautorefname}{Section}   % (2)
% \renewcommand{\defaultspacing}{\doublespacing}  % (3)
% ...
% ...
% ...
% ... add your own customizations here!
\newcommand{\C}{\mathcal{C}}
\delimitershortfall-1sp
\newcommand{\abs}[1]{\lvert#1\rvert}
\newcommand{\norm}[1]{\left\lVert#1\right\rVert}

\newcommand{\dd}[2]{\frac{d#1}{d#2}}
\newcommand{\pd}[2]{\frac{\partial#1}{\partial#2}}

\newcommand{\tang}{\boldsymbol{\tau}}
\newcommand{\normal}{\mathbf{n}}

\DeclareMathOperator{\Div}{div}

\newcommand{\egac}{E_\textrm{GAC}}
\newcommand{\eacwe}{E_\textrm{ACWE}}
\newcommand{\egcs}{E_\textrm{GCS}}

\newcommand{\dx}{\,d\Omega}
\newcommand{\ind}{\mathbf{1}}

\DeclareMathOperator*{\argmin}{arg\,min}

\newcommand{\ip}[2]{\langle #1, #2 \rangle}

%   FRONTMATTER  %%%%%%%%%%%%%%%%%%%%%%%%%%%%%%%%%%%%%%%%%%%%%%%%%%%%%%%%%%%%%%
%
%   Title page, committee page, copyright declaration, abstract,
%   dedication, acknowledgements, table of contents, etc.
%
%   If your research requires an Ethics Statement, download one from the
%   SFU library website and uncomment the appropriate lines below.
%

\begin{document}
\frontmatter
\maketitle{}
\clearpage

%\begin{abstract}
%	This is a blank document from which you can start writing your thesis.
%\end{abstract}
%
%
%\begin{dedication}
%	This is an optional page.
%\end{dedication}
%
%
%\begin{acknowledgements}
%	This is an optional page.
%\end{acknowledgements}

\addtoToC{Table of Contents}%
\tableofcontents%
\clearpage

%\addtoToC{List of Tables}%
%\listoftables%
%\clearpage
%
%\addtoToC{List of Figures}%
%\listoffigures%
%\clearpage





%   MAIN MATTER  %%%%%%%%%%%%%%%%%%%%%%%%%%%%%%%%%%%%%%%%%%%%%%%%%%%%%%%%%%%%%%
%
%   Start writing your thesis --- or start \include ing chapters --- here.
%

\mainmatter%

\setkeys{Gin}{draft}\setkeys{Gin}{draft=false}

\chapter{Introduction}
The aim of this project is to present to the reader several core techniques relevant to image segmentation. At the top level is modelling the image segmentation problem and producing a meaningful mathematical expression. Each of the models we will review is designed around particular image characteristics, whether it is edge-based, where we have the geodesic active contours (GAC) model, or region-based with the active contours without edges (ACWE) model. The mathematical model, or ``energy'', is then minimized to give the desired effect. For image segmentation, that may be wrapping a contour along the edges of an object in a photo, or assigning membership to pixels to divide an image into distinct regions. 

After deriving a reasonable segmentation model, the task then turns to computing a minimizer. A standard minimization technique is gradient descent evolution of the Euler-Lagrange equation. This PDE-based technique is extremely general and can be applied with success to each of the segmentation models we consider. However when the model is convex, powerful algorithms from convex optimization may be more efficient. Along this direction, we will present a unified globally convex segmentation (GCS) model which is an amalgam of the GAC and ACWE model. We will compute solutions from the Euler-Lagrange equation and also by the Split Bregman method, an extremely efficient convex optimization algorithm for $\ell_1$-minimization problems.

Finally, as is the case in any image processing tasks, we have to ask if the practical performance matches the theoretical predictions. To this end, numerical examples will be shown throughout, but with the majority being in \Cref{ch:gcs} where we have the unified model and two minimization algorithms matched up against each other.


Lastly, we also point out that there is but one more section in this introduction and it is a standardized list of notation used throughout. So print it out, open an extra copy on your device (or don't), and read on to the next chapter. 

\newpage
\section{Notation}
\begin{itemize}
	\item $f$: the initial/given 2-D image.
	
	\item $g$: edge indicator function.
	
	\item $\Omega$: image domain.
	
	\item $\Sigma, \Sigma_1, \Sigma_2$ etc.: subsets of the image domain $\Omega$.
	
	\item $\C$: curve/contour, typically closed with well-defined inside and outside. This may also represent a collection of (closed) contours, $\C = \cup_{k \in \mathcal{I}} \C_k$.
	
	\item $\phi$: level set function whose zero level set represents contour $\C$ at time $t$, i.e. $\C(t) = \{(x,y) \in \Omega \mid \phi(x,y,t) = 0\}$. 
	
	\item $t$: time parameter. 
	
	\item $s$: arc length parameter.
	
	\item $H$: Heaviside function
	
	\item $\kappa$: curvature. 
	
	\item $\lambda, \theta, \varepsilon, \rho$: regularization parameters.
	
	\item $\mathbf{1}_\Sigma$: indicator/characteristic function of the subset $\Sigma$, i.e. $\mathbf{1}_\Sigma(x) = 1 \iff x \in \Sigma$.
	
\end{itemize}

\chapter{Geodesic Active Contours}
\label{ch:gac}
The first segmentation model we review is the geodesic active contours (GAC) model from Caselles, Kimmel, and Sapiro \cite{caselles1997geodesic}. Their model of segmentation gives (near) automaticatic selection of an object or objects from an image once the initial contour $\C_0$ is prescribed. To get an active contour $\C = \C(t)$ with $\C(0) = \C_0$, they argued for the minimization of the energy 
\begin{align}
E_\mathrm{GAC}(\C) = \int^{L(\C)}_0 g\left( \abs{\nabla f } \right) \, ds,
\label{eq:egac}
\end{align}
where 
\begin{align*}
f&: \textrm{initial image},
\\
L(\C) &: \textrm{the length of the contour $\C$,} 
\\ 
g(\xi)&: \textrm{an edge indicator function}.
\end{align*}
The energy $E_\textrm{GAC}$ may be interpreted as a weighted arc length. If $g(\xi) = 1$, we would recognize $\int_\C  ds$ as standard Euclidean arc length. The idea is that by designing $g$ is be small near edges and large over homogeneous regions, one would expect the contour $\C$ to be drawn to and remain at object edges when minimizing $E_\textrm{GAC}$. As examples, one may set $g$ as 
\begin{align*}
g_1(\xi) 
= \frac{1}{1 + \alpha\xi^2},
\quad\text{or}\quad 
g_2(\xi) 
= \exp(-\beta\xi^2).
\end{align*}
Notice that as $\abs{\xi} \rightarrow \infty$, $g_1,g_2 \rightarrow 0$.  

In the next sections, we will derive the Euler-Lagrange equation and discuss its discretization for minimizing $\egac$. Numerical examples will demonstrate some of the strengths and weaknesses of this model. One notable weakness of this energy that we point to here is that by design we must be seeking a local minimum. Unfortunately, the global minimum of zero is obtained only when the contour contracts to a point and vanishes. We will revisit this issue when reviewing the numerical examples.


\section{Euler-Lagrange equation to the GAC model} 
A bit of setup and a quick review of vector calculus will go a long ways for this derivation. Let $\C: [0,1]\rightarrow \mathbf{R}^2$ be a parametrized curve, $\C = \C(p)$. We then relate the arc length element to the parameter $p$ as $ds = \abs{C'(p)} dp$. We also recognize the unit tangent vector, $\tang$, unit (inward) normal, $\normal$, and curvature, $\kappa$, as 
\begin{align*}
\tang 
= \frac{\C'}{\abs{\C'}},\quad 
\normal 
= \frac{ \tang '}{\abs{\tang'}},
\quad\text{and}\quad 
\kappa = \frac{ \abs{\tang'} }{ \abs{ \C' } },
\end{align*}
with the prime notation denoting differentiation w.r.t. $p$. We also assume $\C(0) = \C(1)$. 

Rewriting $\egac$ as 
\begin{align*}
\egac = \int^{L(\C)}_0 g \, ds 
= \int^1_0 g(\C) \abs{\C'(p)} \, dp ,
\end{align*} 
we can compute its first variation: 
\begin{align*}
\dd{}{\gamma} \int^1_0 g(\C + \gamma h) \abs{\C' + \gamma h'} \, dp \bigg\rvert_{\gamma = 0}
&=  \int^1_0 \nabla g(\C) \cdot h \abs{\C'} + g(\C) \frac{\C'}{\abs{\C'}} \cdot h' \,dp 
\\
&=\int^1_0 \nabla g(\C) \cdot h \abs{\C'} + g(\C) \tang \cdot h' \,dp 
\\
&=\int^1_0 \nabla g \cdot h \abs{\C'} - ( g \tang)' \cdot h \, dp
\\
&=\int^1_0 \nabla g \cdot h \abs{\C'} - (\nabla g \cdot \C')(\tang \cdot h) - g\tang ' \cdot h \, dp
\\
&=\int^1_0 \nabla g \cdot h \abs{\C'} - (\nabla g \cdot \tang \abs{\C'} )(\tang \cdot h) - g\abs{\tang'} \normal \cdot h \, dp
\\
&=\int^1_0 \big[ \nabla g  - (\nabla g \cdot \tang )\tang \big] \cdot h \abs{\C'} - g\kappa \abs{\C'} \normal \cdot h \, dp
\\
&=\int^1_0 \big[\nabla g \cdot \normal \normal \big] \cdot h \abs{\C'} - g\kappa  \normal \cdot h \abs{\C'} \, dp
\\
&=\int^1_0 \big[ \nabla g \cdot \normal - g\kappa \big] \normal  \cdot h \abs{\C'} \, dp.
\\ 
\end{align*}
This gives the gradient descent equation $\C_t = \left( g(\C) \kappa - \nabla g(\C) \cdot \normal \right) \normal $.
The level set formulation with level set function $\phi = \phi(x, t)$ is
\begin{align}
\begin{split} 
\phi_t 
&= \left(
g( \abs{\nabla I } ) \Div\left( \frac{\nabla \phi}{\abs{\nabla \phi}} \right)
	+  \nabla g(  \abs{\nabla I } )\cdot  \frac{\nabla \phi}{\abs{\nabla \phi}}
\right) \abs{ \nabla \phi }
\\
&= \abs{\nabla \phi} \Div\left( 
g(  \abs{\nabla I } ) \frac{\nabla \phi}{\abs{\nabla \phi}}
\right) .
\end{split}
\label{eq:gac_ls}
\end{align}

\section{Numerical discretization}
\label{sec:gac_num_disc}
To discretize \eqref{eq:gac_ls}, we use the semi-implicit Gauss-Seidel  numerical scheme proposed by Aubert and Vese \cite{aubert1997variational} and used to good effect by Chan and Vese with their ACWE model \cite{chan2001active} rather than using explicit forward Euler with second order centred differences as originally suggested\footnote{Although to be fair, it was suggested for ease of implementation as a proof of concept. For us, the same semi-implicit numerical scheme can and will be used again in \Cref{ch:acwe,ch:gcs} so it makes sense to give a complete presentation once and be able to reuse essentially the same code three times.} in \cite{caselles1997geodesic}.

Define the discrete differential operators: $D_x^+ u_{ij} = (u_{i+1,j} - u_{ij})/h$, $D^0_x = (u_{i+1,j} - u_{i-1,j})/(2h)$,
$D^-_x u_{ij} = (u_{ij} - u_{i-1,j})/h$, and similarly for $D^+_y, D^0_y, D^-_y$.  We will also let $\abs{D^0 \phi^n} = \sqrt{(D^0_x \phi^n_{ij})^2 + (D^0_y \phi^n_{ij})^2 } $ and let $\varepsilon > 0$ be a small regularization parameter. 
Then discretize \eqref{eq:gac_ls} as 
\begin{align*}
\frac{\phi^{n+1}_{ij} - \phi^n_{ij}}{\Delta t} 
&= \abs{D^0 \phi^n}
\left(
D^-_x \left( \frac{g_{ij}D^+_x \phi_{ij}^{n+1}}{\sqrt{ (D^+_x \phi^n_{ij})^2 + (D^0_y \phi^n_{ij})^2 + \varepsilon^2}}
\right) 
+ D^-_y \left(  \frac{g_{ij}D^+_y \phi_{ij}^{n+1}}{\sqrt{ (D^0_x \phi^n_{ij})^2 + (D^+_y \phi^n_{ij})^2  + \varepsilon^2}}
\right)
\right)
\\
&= 
\abs{D^0 \phi^n}\frac{ g_{ij}/h^2 }{\sqrt{ \frac{1}{h^2}(\phi^n_{i+1j} -\phi^n_{ij} )^2 + \frac{1}{4h^2} (\phi^n_{i,j+1} - \phi^n_{i,j-1})^2 + \varepsilon^2}}
(\phi^{n+1}_{i+1,j} - \phi^{n+1}_{ij})
\\
&\quad-\abs{D^0 \phi^n}\frac{ g_{i-1,j}/h^2 }{\sqrt{ \frac{1}{h^2}(\phi^n_{ij} -\phi^n_{i-1,j} )^2 + \frac{1}{4h^2} (\phi^n_{i-1,j+1} - \phi^n_{i-1,j-1})^2 + \varepsilon^2}}
(\phi^{n+1}_{ij} - \phi^{n+1}_{i-1,j}) 
\\
&\quad+\abs{D^0 \phi^n}\frac{ g_{ij}/h^2 }{\sqrt{ \frac{1}{4h^2}(\phi^n_{i+1,j} -\phi^n_{i-1,j} )^2 + \frac{1}{h^2} (\phi^n_{i,j+1} - \phi^n_{ij})^2 + \varepsilon^2}}
(\phi^{n+1}_{i,j+1} - \phi^{n+1}_{ij}) 
\\
&\quad-\abs{D^0 \phi^n}\frac{ g_{i,j-1}/h^2 }{\sqrt{ \frac{1}{4h^2}(\phi^n_{i+1,j-1} -\phi^n_{i-1,j-1} )^2 + \frac{1}{h^2} (\phi^n_{ij} - \phi^n_{i,j-1})^2 + \varepsilon^2}}
(\phi^{n+1}_{ij} - \phi^{n+1}_{i,j-1}) 
\\
&\eqqcolon \abs{D^0 \phi^n} \big[ a_1(\phi^{n+1}_{i+1,j} - \phi^{n+1}_{ij})
- a_2(\phi^{n+1}_{ij} - \phi^{n+1}_{i-1,j}) 
+ a_3(\phi^{n+1}_{i,j+1} - \phi^{n+1}_{ij}) 
- a_4(\phi^{n+1}_{ij} - \phi^{n+1}_{i,j-1})
\big]
\\
&= \abs{D^0 \phi^n} 
\big[ a_1\phi^{n+1}_{i+1,j} 
+ a_2\phi^{n+1}_{i-1,j} 
+ a_3\phi^{n+1}_{i,j+1} 
+ a_4\phi^{n+1}_{i,j-1}
- (a_1 + a_2 + a_3 + a_4) \phi^{n+1}_{ij}
\big].
\end{align*}
Hence 
\begin{align*}
\big[ 
\underbrace{1 + \Delta t\abs{D^0 \phi^n} (a_1 + a_2 + a_3 + a_4) \big]
}_{\eqqcolon a_0} \phi^{n+1}_{ij} 
= \phi^n_{ij} + \Delta t\abs{D^0 \phi^n} (a_1\phi^{n+1}_{i+1,j} 
+ a_2\phi^{n+1}_{i-1,j} 
+ a_3\phi^{n+1}_{i,j+1} 
+ a_4\phi^{n+1}_{i,j-1}).
\end{align*}
The Gauss-Seidel iterations would be (if working through $(i,j)$ in the usual order)
\begin{align*}
\phi^{n+1}_{ij} 
= \frac{1}{a_0} 
\left( \phi^n_{ij} + \Delta t \abs{D^0 \phi^n}
\left( a_1 \phi^n_{i+1,j} + a_2 \phi^{n+1}_{i-1,j} + a_3 \phi^{n}_{i,j+1} + a_4 \phi^{n+1}_{i,j-1}
\right)
\right).
\end{align*}
One final note with regards to this numerical discretization. It is suggested to alternate the discretization of the divergence term, eg. applying $D^+_x$ on the outside and $D^-_x$ inside, and with various combinations with $D^+_y$, $D^-_y$ as well. Our testing shows this alternating reduces asymmetries in our solutions to the GAC model but produces no observable differences with our later segmentation models.


\section{Examples and discussion} 
In this section, parameters are $\Delta t = 0.05$, $h = 1$, and $\varepsilon = 10^{-6}$, unless otherwise stated.
The edge indicator function used is 
\begin{align}
g = \frac{1}{1 + \abs{\nabla \widetilde f}^2 },
\label{eq:edge_indicator}
\end{align}
where $\widetilde f$ is a Gaussian-smoothed version of the image $f$. Modification of the GAC model to include a constant velocity, $c$, in the normal direction, 
\begin{align}
\phi_t 
= \left( g ( c + \kappa) + \nabla g \cdot \frac{\nabla \phi}{\abs{\nabla \phi}} \right) \abs{\nabla \phi}
= 
\abs{\nabla \phi} \Div\left( g \frac{\nabla \phi}{\abs{\nabla \phi}} \right) 
+ cg \abs{\nabla \phi},
\label{eq:gac+c}
\end{align}
and is explored as well. More detail on this below.


Our first series of tests is on a synthetic image with 8 objects (squares) with which we run four scenarios starting from two different initial contours. This is shown in \Cref{fig:grid}. In the top row is the first two scenarios with the initial contour  beinng a single, giant circle. From here we run two cases. With no added constant normal velocity, i.e. $c = 0$, we get a bounding box on our eight squares (top row middle). This is a local minimum and the contour does not drive into the gaps. Supplied with an extra push, i.e. setting $c = -1$, the segmentation algorithm picks out all eight squares (top right).
	
In the second row, we have multiple circles forming the initial contour, with two of the circles enclosing one square each and the centre circle feeling a little empty. Setting $c = 0$ (and $c < 0$), the contours contract, causing the centre circle to vanish and the other two to wrap tightly around their square. Setting $c = 1$, the contours expand, missing two squares.
	
We find the model adheres to edges nicely. Whenever the contour locates an edge, it stays there. However, even in our limited testing it is clear that this method requires significant high level input in at least two ways. First, the ``right'' initial contour is needed to generate the desired result. This may require, for every image, a user having to drawing a rough contour around all objects before letting the algorithm take over. Second would be to specify $c$. Not only does nonzero values of $c$ produce drastically different results from $c = 0$, but it also drove the solution to steady state faster. Without a suitable value of $c$, this method is impractical. In some of our cases, the total number of iterations was reduced by a factor of over $100$.
	
\begin{figure}[htb!]
	\centering
	\begin{minipage}{0.31\textwidth}
				\includegraphics[width=\textwidth]{grid1}
	\end{minipage}\,
	\begin{minipage}{0.31\textwidth}
		\includegraphics[width=\textwidth]{grid1ssc0}
	\end{minipage}\,
	\begin{minipage}{0.31\textwidth}
		\includegraphics[width=\textwidth]{grid1sscm1}
	\end{minipage}
	\begin{minipage}{0.31\textwidth}
		\includegraphics[width=\textwidth]{grid2}
	\end{minipage}\,
	\begin{minipage}{0.31\textwidth}
		\includegraphics[width=\textwidth]{grid2ssc0}
	\end{minipage}\,
	\begin{minipage}{0.31\textwidth}
		\includegraphics[width=\textwidth]{grid2sscp1}
	\end{minipage}
	
	\caption{(Left) Initial setups. (Middle) Steady state solution with $c = 0$. (Top right) Steady state solution with $c = -1$. (Bottom right) Steady state solution with $c = 1$.}
	\label{fig:grid}
\end{figure}

In closing this chapter, we recognize that while one example is by no means a comprehensive study on this model, we have identified a key issues we fully expect the next segmentation models to improve upon, namely sensitivity to the initial condition and reliance on high level input.
	



\chapter{Active Contours Without Edges}
\label{ch:acwe}
The previous section we reviewed the GAC model and outlined its shortcomings. Perhaps the key deficiency of that model is its sensitivity to the initial conditions and its heavy reliance on user directed input to generate the desired result. 

The active contours without edges (ACWE) model developed by Chan and Vese \cite{chan2001active} resolves these issues by proposing an energy based upon 2 fitting terms (for 2-phase segmentation). The energy they sought to minimize is 
\begin{align}
\begin{split} 
\eacwe(\C, c_1, c_2; \lambda)
&= \textrm{ Length($\C$) } 
+ \lambda \int_{\Sigma} (c_1 - f )^2 \dx
+ \lambda \int_{\Omega \setminus \Sigma} (c_2 - f )^2 \dx
\\
&= \textrm{ Perimeter($\Sigma$) } 
+ \lambda \int_{\Sigma} (c_1 - f )^2 \dx
+ \lambda \int_{\Omega \setminus \Sigma} (c_2 - f )^2 \dx,
\end{split}
\label{eq:acwe_e}
\end{align}
where $\Sigma \subset \Omega$ is the region enclosed by the contour $\C$.
The last two terms of $\eacwe$ are the fitting terms we were referring to and within them two key quantities, $c_1, c_2 \in \mathbf{R}$, are introduced. They are best understood with a simple piecewise constant image with 2 homogeneous regions, see Figure \ref{fig:fitting}. Suppose the gray levels of the image are 0 and 1, with 1 being maximum intensity. Then with the contour as in Figure \ref{fig:gull}, $c_1 = 1$ and $c_2 = 0$ would minimize the fitting terms as each integral will evaluate to zero. However, with any other contour, eg. \Cref{fig:tiger} or \Cref{fig:mouse}, one or both fitting terms will be positive regardless of how one chooses $c_1$ and $c_2$. Consequently, minimizing $\eacwe$ drives the contour towards the ``best fit'' and does so without relying on image gradients.

In the next sections we will derive an Euler-Lagrange equation to the ACWE model, provide a numerical discretization, and solve to steady state. Strengths and weaknesses of the model will be discussed.

\begin{figure}
	\centering
	\begin{subfigure}[b]{0.31\textwidth}
		\includegraphics[width=\textwidth]{acwe_0e}
		\caption{Minimizes fitting energy}
		\label{fig:gull}
	\end{subfigure}
	~ %add desired spacing between images, e. g. ~, \quad, \qquad, \hfill etc. 
	%(or a blank line to force the subfigure onto a new line)
	\begin{subfigure}[b]{0.31\textwidth}
		\includegraphics[width=\textwidth]{acwe_1e}
		\caption{Positive fitting energy}
		\label{fig:tiger}
	\end{subfigure}
	~ %add desired spacing between images, e. g. ~, \quad, \qquad, \hfill etc. 
	%(or a blank line to force the subfigure onto a new line)
	\begin{subfigure}[b]{0.31\textwidth}
		\includegraphics[width=\textwidth]{acwe_2e}
		\caption{Positive fitting energy}
		\label{fig:mouse}
	\end{subfigure}
	\caption{Visual representation of the fitting energy terms in \eqref{eq:acwe_e}}
	\label{fig:fitting}
\end{figure}


\section{Euler-Lagrange equation to the ACWE model}
To be clear, minimizing $\eacwe$ is very hard as there is $\C$ (or equivalently $\Sigma$), $c_1$, and $c_2$ to consider. Rather than simultaneous minimization w.r.t. $\C$, $c_1$, and $c_2$, we will follow the clever alternating minimization scheme suggested in \cite{chan2001active}. The procedure will be to optimize first w.r.t. $c_1$ and $c_2$ with $\C$ fixed, then with $c_1$ and $c_2$ determined, minimize w.r.t. $\C$.

First with fixed $\C$, it is then elementary calculus to determine the optimal values of $c_1$ and $c_2$: 
\begin{align}
0 = \pd{\eacwe}{c_1}  = 2\lambda \int_{\Sigma} (c_1 - f ) \dx
&\implies 
c_1 = \frac{1}{\abs{\Sigma}} \int_{\Sigma} f \dx,
\label{eq:c1}
\\
0 = \pd{\eacwe}{c_2}  = 2\lambda \int_{\Omega\setminus \Sigma} (c_2 - f )\dx
&\implies 
c_2 = \frac{1}{\abs{\Omega\setminus\Sigma}} \int_{\Omega\setminus \Sigma} f \dx.
\label{eq:c2}
\end{align}
Here we have used $\abs{\cdot}$ to denote the area of a set. Next is the minimization of $\eacwe(\cdot, c_1, c_2; \lambda)$, i.e. $\min_{\Sigma \subset \Omega} \eacwe(\Sigma)$. Rewriting in terms of a level set function, $\phi$, to transfer the dependence on $\Sigma$ directly into the integrand, we have 
\begin{align}
\textrm{Perimeter}(\Sigma) 
&= \int_{\Omega} \abs{\nabla \mathbf{1}_\Sigma(x)} \dx
= \int_{\Omega} \abs{ \nabla H(\phi(x)) } \dx,
\label{eacwe1}
\\
\int_{\Sigma} (c_1 - f(x) )^2 \, d\Omega 
&= \int_{\Omega} (c_1 - f(x))^2 H(\phi(x)) \, d\Omega
\\
\int_{\Omega\setminus\Sigma} (c_2 - f(x))^2 \dx 
&=
\int_{\Omega} (c_2 - f(x))^2 (1 - H(\phi(x))) \dx
\end{align}
Looking ahead, it may be necessary to differentiate the Heaviside and the delta function This technicality is averted by subsituting with a regularized version: $H_\rho \in C^2$, and $H_\rho \rightarrow H$ as $\rho\rightarrow 0$ and denote $\delta_\rho = H'_\rho$. Thus the regularized energy (which we will continue to call $\eacwe$) is
\begin{align*}
\eacwe(\phi) 
&= \int_{\Omega } \abs{\nabla H_\rho(\phi(x))} \dx 
+ \lambda\int_{\Omega} (c_1 - f(x))^2 H_\rho(\phi(x)) 
+ (c_2 - f(x))^2 (1 - H_\rho(\phi(x))) \dx
\\
&= \int_{\Omega } \delta_\rho(\phi(x)) \abs{\nabla \phi(x)} \dx 
+ \lambda\int_{\Omega} (c_1 - f(x))^2 H_\rho(\phi(x)) 
+ (c_2 - f(x))^2 (1 - H_\rho(\phi(x))) \dx,
\end{align*}
and its first variation
\begin{align*}
\dd{}{\gamma}\eacwe(\phi + \gamma h) \bigg\rvert_{\gamma = 0}
&= \int_{\Omega} \delta'_\rho(\phi) \abs{\nabla \phi} h + \delta_\rho(\phi) \frac{\nabla \phi}{\abs{\nabla \phi}} \cdot \nabla h \dx
+ \lambda\int_{\Omega} \underbrace{\left[(c_1 - f)^2 - (c_2 - f)^2 \right]}_{=r(x)} \delta_\rho(\phi) h \dx 
\\
&= \int_{\Omega } \delta'_\rho \abs{\nabla \phi} h
- \Div\left(\delta_\rho\frac{\nabla \phi}{\abs{\nabla \phi}}  \right)  h
\dx 
+ \int_\Gamma h\delta_\rho\frac{\nabla \phi}{\abs{\nabla \phi}} \cdot \normal \,d\Gamma 
+ \lambda\int_{\Omega} r \delta_\rho h \dx
\\
&= \int_{\Omega} \delta_\rho\left[ 
- \Div\left(\frac{\nabla \phi}{\abs{\nabla \phi}} \right) + \lambda r
\right] h\dx 
+ \int_\Gamma h\delta_\rho\frac{\nabla \phi}{\abs{\nabla \phi}} \cdot \normal \,d\Gamma. 
\end{align*}
The gradient descent evolution is thus 
\begin{align}
\phi_t 
= \delta_\rho(\phi) \left[ 
\Div\left(\frac{\nabla \phi}{\abs{\nabla \phi}} \right) - \lambda (c_1 - f)^2 + \lambda (c_2 - f)^2 
\right]
\label{eq:acwe_el}
\end{align}
with boundary condition $\delta_\rho 
\frac{\nabla \phi}{\abs{\nabla \phi}} \cdot \normal = 0$.


\section{Numerical discretization}
\label{sec:3.2}
As was mentioned, the numerical scheme we apply to \eqref{eq:acwe_el} will largely be the same as the scheme detailed for the GAC model in \Cref{sec:gac_num_disc}. The difference are $\delta_\rho(\phi^n_{ij})$ in place of $\abs{D^0 \phi^n}$, $g_{ij} = 1$ for all $(i,j)$, and the inclusion of the new fitting terms.

The fitting terms are straightforward to handle as they have no explicit $\phi$-dependence. Let $r(x, c_1, c_2) = [(c_1 - f(x))^2 - (c_2 - f(x))^2 ]$. At the beginning of each timestep, update $c_1$ and $c_2$ according to \eqref{eq:c1} and \eqref{eq:c2} and set
\begin{align*}
r_{ij}^n = \big[ (c_1^n - f_{ij})^2 - (c_2^n - f_{ij})^2 \big].
\end{align*}
The expression for $\delta_\rho(\phi)$ follows straight from our choice of $H_\rho(\phi)$, which we set as in \cite{chan2001active}:
\begin{align*}
H_\rho(z)
 = \frac{1}{2} 
\left(1 + \frac{2}{\pi}\arctan
\left(\frac{z}{\rho} 
\right) \right)
\implies 
\delta_\rho(z)  
=  \frac{1}{\pi}\frac{\rho}{\rho^2  + z^2}.
\end{align*}
Very important to note that this choice of $H_\rho$ is ``global'', unlike the typical choices which are local and nonzero only over a small neighbourhood. The authors argue against a compactly supported regularization of $H$, saying that because the energy is non-convex and may admit local minima, a global $H_\rho$ is more likely to compute a global minimizer and allows the model to automatically detect interior contours (both of which they observed in practice). We also set $\rho = h$ as recommended, i.e. use $\delta_h(\phi_{ij})$. The fact that $H_\rho$ is global will also play a significant role in the next chapter.

All together, this gives us the numerical scheme
\begin{align*}
\phi^{n+1}_{ij} 
= \frac{1}{a_0} 
\left( \phi^n_{ij} + \Delta t \delta_h(\phi_{ij}^n)
\left( a_1 \phi^n_{i+1,j} + a_2 \phi^{n+1}_{i-1,j} + a_3 \phi^{n}_{i,j+1} + a_4 \phi^{n+1}_{i,j-1} 
- \lambda r_{ij}^n
\right)
\right),
\end{align*}
with $a_1, a_2, a_3, a_4$  as given in \Cref{sec:gac_num_disc}, $g_{ij} = 1$, and $a_0 = 1 + \Delta t \delta_h(\phi_{ij}^n)(a_1 + a_2 + a_3 + a_4)$.

\section{Examples and discussion}
In this section, parameters are $\Delta t = 0.05$, $h = 1$, and $\varepsilon = 10^{-6}$, $\lambda = 10$, unless otherwise stated.

We start with a repeat of the eight squares example from the previous section, but this time with Gaussian white noise added to the initial image, see \Cref{fig:grid_acwe}. We tested five different initial contours and all reached the desired segmentation. The ability to easily handle noisy images and detect objects without significant user input certainly puts the ACWE model ahead of the GAC model.

\begin{figure}[htb!]
	\centering
	\begin{minipage}{0.31\textwidth}
		\includegraphics[width=\textwidth]{acgrid1}
	\end{minipage}%
	\begin{minipage}{0.31\textwidth}
		\includegraphics[width=\textwidth]{acgrid5}
	\end{minipage}%
	\begin{minipage}{0.31\textwidth}
		\includegraphics[width=\textwidth]{acgrid4}
	\end{minipage}
	\begin{minipage}{0.31\textwidth}
		\includegraphics[width=\textwidth]{acgrid3}
	\end{minipage}%
	\begin{minipage}{0.31\textwidth}
		\includegraphics[width=\textwidth]{acgrid2}
	\end{minipage}%
	\begin{minipage}{0.31\textwidth}
		\includegraphics[width=\textwidth]{acgridsol}
	\end{minipage}
	
	\caption{Only the bottom right  is a plot of the final state. The five others are different initial contours.}
	\label{fig:grid_acwe}
\end{figure}
However, this method is not guaranteed to find the global minimizer. In \Cref{fig:target_acwe} is one such example. Beginning from 3 different contours, we ended at 3 slightly different results. This issue will be addressed in the next chapter with the introduction of a convex segmentation model with which one can be guarantee to find a global minimum regardless of the starting contour. In addition, a fast algorithm from convex minimization will also be presented and shown to offer quality segmentation results at much lower computation costs.
\begin{figure}[htb!]
	\centering
	\begin{minipage}{0.31\textwidth}
		\includegraphics[width=\textwidth]{brainacout0}
	\end{minipage}%
	\begin{minipage}{0.31\textwidth}
		\includegraphics[width=\textwidth]{brainaccir0}
	\end{minipage}%
	\begin{minipage}{0.31\textwidth}
		\includegraphics[width=\textwidth]{brainacbub0}
	\end{minipage}
	\begin{minipage}{0.31\textwidth}
		\includegraphics[width=\textwidth]{brainacout}
	\end{minipage}%
	\begin{minipage}{0.31\textwidth}
		\includegraphics[width=\textwidth]{brainaccir}
	\end{minipage}%
	\begin{minipage}{0.31\textwidth}
		\includegraphics[width=\textwidth]{brainacbub}
	\end{minipage}
	\caption{(Top) Initial image with different starting contours. (Bottom) Steady state solutions.}
	\label{fig:target_acwe}
\end{figure}


\chapter{Globally Convex Segmentation}
\label{ch:gcs}
Numerical experiments in the two models, GAC and ACWE, show that there is certainly room for improvement from the modelling perspective. Of particular concern are the models knack for finding and getting stuck at local minima (by design, in case of the GAC model), and therefore the solution attained is dependent on the initial contour. These issues are address by Chan, Esedo\={g}lu, and Nikolova \cite{chan2006algorithms}, and their ideas further refined by Bresson et al. \cite{bresson2007fast}, in what they referred to as the convexification of the ACWE model and the unification of two models we have considered.

Before going in depth, we should explain that in this chapter our viewpoint on the segmentation problem shifts somewhat. In the previous models, the viewpoint was mainly on an active contours and its evolution. However in the GCS model, the viewpoint is centred about indicator functions (of sets). Keep in mind that for 2-phase segmentation, any subset $\Sigma \subset \Omega$, and equivalently the corresponding indicator function $\ind_\Sigma$, defines a segmentation of the image domain. In what follows, the goal will arriving at an indicator function $u(x) = \ind_\Sigma(x)$, and the piecewise constant solution $\widetilde u(x) = c_1 \ind_\Sigma(x) + c_2 (1 - \ind_\Sigma(x))$. In other words, the problem may instead be framed as optimizating over functions that take only two values to find the best approximation to a given image $f$.

But first it is important to understand the nature of the non-convexity in the ACWE model. Restating the optimization problem in terms of indicator functions, we have 
\begin{align}
\min_{\substack{\Sigma\subset\Omega \\ 
		u(x) = \ind_\Sigma(x)}} 
\left\{\eacwe(u, c_1, c_2; \lambda)
= \int_{\Omega} \abs{\nabla u} \dx 
+ \int_\Omega u(c_1 - f)^2   + (1-u)(c_2 - f)^2 \dx 
\right\}.
\label{eq:eacwe2}
\end{align}
Observe that the function set we are optimizing over is not convex. For instance suppose $\Sigma_1, \Sigma_2 \subset \Omega$, $\Sigma_1 \cap \Sigma_2 = \emptyset$, $\Omega \setminus (\Sigma_1 \cup \Sigma_2) \neq \emptyset$ and set $u_1 = \ind_{\Sigma_1}$ and $u_2 = \ind_{\Sigma_2}$. Then any convex combination $w = ku_1 + (1-k)u_2$, $k \in (0,1)$, would be a function that takes on three values. 

In the next section are two key theorems which will allow that constraint to be relaxed, i.e. allow $u$ to take on the continuum of values between 0 and 1, and lead to an equivalent but convex minimization problem.

\section{A unified GAC \& ACWE convex segmentation model}
We first start the key observation that in choosing a non-compactly support $\delta_\epsilon$ in the gradient descent evolution \eqref{eq:acwe_el}, the following will have the same steady state solutions: 
\begin{align*}
\phi_t = \Div\left(\frac{\nabla \phi}{\abs{\nabla \phi}} \right) 
- \lambda (c_1 - f)^2 + \lambda (c_2 - f)^2 .
\end{align*}
This in turn is the gradient descent equation the following energy:
\begin{align}
E(\phi, c_1, c_2) 
= \int_\Omega \abs{\nabla \phi} \dx 
+ \lambda \int_\Omega \big[ (c_1 - f)^2 - (c_2 - f)^2 \big] \phi \dx ,
\label{eq:mod_acwe}
\end{align}
which gets us to the first of two key theorems from \cite{chan2006algorithms}.

\begin{thm}
	For any given fixed $c_1, c_2 \in \mathbf{R}$, a global minimizer for $\eacwe(\cdot, c_1, c_2; \lambda)$ can be found by carrying out the following convex minimization: 
	\begin{align}
	\min_{0\leq u \leq 1}\left\{
		\widetilde E_{\textrm{ACWE}} (u, c_1, c_2; \lambda)
		=
		\int_{\Omega} \abs{\nabla u} \dx 
	+ \lambda\int_{\Omega} \big[ (c_1 - f(x) )^2 - (c_2 - f(x))^2 \big] u(x) \dx
	\right\}
	\label{eq:thm1}
	\end{align}
	and then setting $\Sigma = \{ x \mid u(x) \geq \mu \}$ for a.e. $\mu \in [0, 1]$.
	\label{thm:1}
\end{thm}
The theorem above links the two energies, $\eacwe$ and $\widetilde E_\textrm{ACWE}$,
and guarantees that any global minimum of \eqref{eq:mod_acwe} is only a thresholding step away from a global minimum of the ACWE model. The next theorem is but one way to tackle \eqref{eq:thm1}.

\begin{thm}
	Let $r(x) \in L^\infty(\Omega)$. Then the convex, constrained minimization problem
	\begin{align*}
	\min_{0 \leq u \leq 1} \int_{\Omega} \abs{\nabla u} \dx + \lambda \int_{\Omega} r(x) u \dx 
	\end{align*}
	has the same set of minimizers as the following convex, unconstrained minimization problem:
	\begin{align*}
	\min_u \int_{\Omega} \abs{\nabla u} \dx + \lambda \int_{\Omega} r(x) u + \alpha \nu(u) \dx 
	\end{align*}
	where $\nu(\xi) = \max\{ 0 , 2\abs{\xi - \frac{1}{2} } - 1\}$, provided that $\alpha > \frac{\lambda}{2}\norm{r(x)}_{L^\infty(\Omega)}$.
	\label{thm:2}
\end{thm}
The term $\alpha\nu(u)$ is an exact penalty term \cite{hiriart1993convexI,hiriart1993convexII}. The advantage of this new unconstrained formulation is that it is quite straightforward to derive the Euler-Lagrange equation and solve by gradient descent. But before doing so, we will add one more modification to give a unified globally convex segmentation (GCS) model.

Per Bresson et al. \cite{bresson2007fast}, building on the work of \cite{chan2006algorithms}, they propose minimization of the energy 
\begin{align}
\egcs(u, c_1, c_2; \lambda) = 
\int_{\Omega} g(x)  \abs{\nabla u} \dx 
+ \lambda\int_{\Omega} \big[ (c_1 - f(x) )^2 - (c_2 - f(x))^2 \big] u(x) \dx,
\end{align}
where $g$ is an edge indicator function as in \Cref{ch:gac}. Furthermore, if $u$ is an indicator function, then
\begin{align*}
\egcs(\ind_\Sigma(x), c_1, c_2; \lambda) 
&= \int_{\Omega} g(x)  \abs{\nabla \ind_\Sigma(x)} \dx 
+ \lambda\int_{\Omega} \big[ (c_1 - f(x) )^2 - (c_2 - f(x))^2 \big] \ind_\Sigma(x) \dx
\\
&= \egac(\partial \Sigma) +  \lambda\int_{\Omega} \big[ (c_1 - f(x) )^2 - (c_2 - f(x))^2 \big] \ind_\Sigma(x) \dx
\end{align*}
reduces to the familiar GAC energy but now subject to an ACWE-type fitting energy constraint. Finally, also note that $\egcs$ satisfies statements same as those in Theorem \ref{thm:1} and \ref{thm:2}, but with $\int g\abs{\nabla u}$ in place of $\int \abs{\nabla u}$.

The next section will not be just one, but two numerical algorithms for image segmentation under the GCS model. In addition to the usual Euler-Lagrange equation and gradient descent, we will also discuss application of the Split Bregman method \cite{goldstein2010geometric,goldstein2009split} to this segmentation model.

\section{Minimization of the GCS model: gradient descent and the Split Bregman}
\subsection{Euler-Lagrange for the GCS model}
For this subsection, we are considering the unconstrained minimization problem 
\begin{align*}
\min_u \underbrace{\int_{\Omega } g\abs{\nabla u} \dx 
+ \int_{\Omega} \lambda \overbrace{\big[ (c_1 - f)^2 - (c_2 - f)^2 \big]}^{=r(x)} u + \alpha \nu(u) \dx }_{=\widetilde E_{\textrm{GCS}}(u)}
\end{align*}
($\egcs$ with an exact penalty term, see Theorem \ref{thm:2}).
Fixing $c_1, c_2$ and $\lambda$, we have
\begin{align*}
\dd{}{\gamma} \widetilde E_\textrm{GCS}(u + \gamma h)
\bigg\rvert_{\gamma = 0} 
&= \int_{\Omega} g \frac{\nabla u}{\abs{\nabla u}} \cdot \nabla h
+ \left( \lambda r + \alpha \nu'(u) \right) h \dx 
\\
&= \int_{\Gamma} h g \frac{\nabla u}{\abs{\nabla u}} \cdot \normal  \, d \Gamma - \int_{\Omega } \Div\left( g \frac{\nabla u}{\abs{\nabla u}} \right) - \lambda r - \alpha \nu'(u) \dx ,
\end{align*}
which gives us the descent equation 
\begin{align*}
u_t = \Div\left( g(x)  \frac{\nabla u}{\abs{\nabla u}} \right) - \lambda\big[ (c_1 - f)^2 - (c_2 - f)^2 \big]  - \alpha \nu'(u)
\end{align*}
and boundary condition $g\frac{\nabla u}{\abs{\nabla u}} \cdot \normal = 0$. 

Those familiar with exact penalty functions will note that there is a techical issue with the descent equation. As defined, $\nu(\xi)$ has two points of non-differentiability at $\xi = 0$ and $\xi = 1$. It will be necessary to substitute with a smooth penalty function, $\nu_\varepsilon(\xi)$. In our numerical examples we will be using a smoothed penalty function similar to one analyzed by Pinar and Zenios \cite{pinar1994smoothing}.



\subsection{The Split Bregman method}
In \cite{goldstein2010geometric}, Goldstein, Bresson and Osher applied the Split Bregman method \cite{goldstein2009split} to the GCS image segmentation model with good results. An advantage of using the Split Bregman method is that we are able to solve the minimization problem 
\begin{align}
\min_{0 \leq u \leq 1} \left\{ \egcs  
= \int_{\Omega } g\abs{\nabla u} \dx 
+ \lambda \int_{\Omega} \big[ (c_1 - f)^2 - (c_2 - f)^2\big] u \dx
\right\}
\label{eq:egcs1}
\end{align}
without the need of an exact penalty term or any added regularization of the TV-norm. We will compare and contrast this alternative to the earlier gradient descent approach. 

To derive the Split Bregman formulation, we'll first restate \eqref{eq:egcs1} in discrete form, define the auxiliary variable $d = (d^x, d^y) = (u_x, u_y) = \nabla u$ and introduce a quadratic penalty function to weakly enforce this equality. This gets us 
\begin{align*}
\min_{0\leq u \leq 1, d} 
 \norm{d }_g + \lambda  \langle r, u \rangle
+ \frac{\theta}{2} \norm{ d - \nabla u }_2^2,
\end{align*}
where $\norm{d}_g = \norm{g \nabla u}_1$. For problems such as ours \footnote{For more details, one can refer to the appendix, and \cite{goldstein2009split,goldstein2010geometric}}, the Bregman iteration simplifies to the 2-step procedure 
\begin{align}
(u^{k+1}, d^{k+1}) 
&= \argmin_{0\leq u \leq 1, d}  \norm{d}_g + \lambda \langle r, u \rangle + \frac{\theta}{2} \norm{d - \nabla u - b^k }^2_2 
\\
b^{k+1} &= b^k - (d^k - \nabla u^k) .
\label{eq:sbbij}
\end{align} 
The Split Bregman then breaks up the first step into 2 substeps that may be iterated until convergence, effectively decoupling $u$ and $d$: 
\begin{align}
u^{k+1} &= \argmin_{0\leq u \leq 1} \lambda \langle r, u \rangle
+ \frac{\theta}{2} \norm{d^k - \nabla u - b^k }^2_2,
\label{eq:splitu}
\\
d^{k+1} 
&= \argmin_d \norm{d}_g + \frac{\theta}{2} \norm{d - \nabla u^{k+1} - b^k}^2_2.
\label{eq:splitd}
\end{align}
Each of the substeps can be solved efficiently. We begin by tackling the solution to \eqref{eq:splitd}.

To begin, let's clarify precisely what the vector norm $\norm{\cdot }_g$ is as there is more than one choice. We will use the one referred to as isotropic TV in \cite{goldstein2009split},
\begin{align*}
\norm{ d }_g 
= \sum^N_{i,j = 1} g_{ij} \norm{d_{ij}}_2 
= \sum^{N}_{i,j = 1} g_{ij} \sqrt{ (d^x_{ij} )^2 + (d^y_{ij} )^2 }
\end{align*}
in which case \eqref{eq:splitd} is equivalently 
\begin{align*}
\sum^{N}_{i,j = 1} g_{ij} \sqrt{ (d^x_{ij} )^2 + (d^y_{ij} )^2 }
+ \frac{\theta}{2} \sum^N_{i,j=1} \big[ (d^x_{ij} - \nabla_x u^{k+1}_{ij} - (b^x_{ij})^k)^2 
+ (d^y_{ij} - \nabla_y u^{k+1}_{ij} - (b^y_{ij})^k)^2 
\big]
\eqqcolon \sum^N_{i,j=1} S_{ij}
\end{align*}
For each $(i,j)$ we can optimize by elementary calculus, 
\begin{align*}
0
=\pd{S_{i,j}}{d^x_{i,j}} 
= g_{ij}\frac{d^x_{ij}}{ \norm{d_{ij}}_2}
+ \theta (d^x_{ij} - \nabla_x u^{k+1}_{ij} - (b^x_{ij})^k) 
\iff
\nabla_x u^{k+1}_{ij} + (b^x_{ij})^k  
= d^x_{ij} \left( 1 + \frac{g_{ij}}{\theta \norm{d_{ij}}_2}
\right)
\\
0
=\pd{S_{ij}}{d^y_{i,j}} 
= g_{ij}\frac{d^y_{ij}}{ \norm{d_{ij}}_2}
+ \theta (d^y_{ij} - \nabla_y u^{k+1}_{ij} - (b^y_{ij})^k) 
\iff
\nabla_y u^{k+1}_{ij} + (b^y_{ij})^k 
= d^y_{ij} \left( 1 + \frac{g_{ij}}{\theta\norm{d_{ij}}_2}
\right)
\end{align*}
which combines as 
\begin{align*}
\norm{\nabla u^{k+1}_{ij} + b^k_{ij}}_2^2 = \left(\norm{ d_{ij}}_2 + \frac{g_{ij}}{\theta}
\right)^2
\implies 
\norm{d_{ij}}_2 = \norm{\nabla u^{k+1}_{ij} + b^k_{ij} }_2 - \frac{g_{ij}}{\theta},
\end{align*}
(if solution exists), so
\begin{align*}
d_{ij}^x 
= \frac{ \nabla_x u^{k+1}_{ij} + (b^x_{ij})^k }{ \norm{\nabla u^{k+1}_{ij} + b^k_{ij} }_2} \left( \norm{\nabla u^{k+1}_{ij} + b^k_{ij} }_2 - \frac{g_{ij}}{\theta} \right), 
\end{align*}
and similarly for $d^y_{ij}$. Otherwise, there is one other critical point giving the minimum at $d = 0$; the full solution is typically presented as 
\begin{align}
(d^x)^{k+1} 
= \max\left( \norm{ \nabla u^{k+1} + b^k}_2 - \frac{g}{\theta}, 0 \right) \frac{ \nabla_x u^{k+1} + (b^x)^k}{ \norm{ \nabla u^{k+1} + b^k}_2}
\label{eq:sbdij}
\end{align}
and likewise for $(d^y)^{k+1}$.

The solution to \eqref{eq:splitu} can be found by first differentiating to get $0 = \lambda  +\theta  \nabla \cdot ( d^k - \nabla u - b^k)$ which leads to $\Delta u = \frac{\lambda}{\theta} r + \nabla \cdot (d^k - b^k)$. As the energy is quadratic in $u_{ij}$, either the optimal value is given by this PDE, or if it were outside the interval $[0,1]$, then must take either $0$ or $1$, whichever is nearer the unconstrained minimizer. For example, with centred differencing of the Laplacian, we have 
\begin{align}
\widetilde u_{ij} = \frac{1}{4}\left(
u_{i+1,j} + u_{i-1,j} + u_{i,j+1} + u_{i,j-1} 
- \frac{\lambda}{\theta} r_{ij}
- \nabla_x ((d^x_{ij})^k - (b^x_{ij})^k)
- \nabla_y((d^y_{ij})^k - (b^y_{ij})^k)
\right) 
\label{eq:sbuij}
\end{align}
and 
\begin{align}
u_{ij}  = \max \{ \min\{ \widetilde u_{ij}, 1 \} , 0 \}.
\label{eq:sbuij2}
\end{align}
The full algorithm, to summarize, is
\begin{enumerate}
	\item Threshold $u^k$ to determine $\Sigma^k$, eg. $\Sigma^k = \{x\in\Omega \mid u(x) \geq 0.5 \}$.
	\item Update $c_1, c_2$ by \Cref{eq:c1,eq:c2} with $\Sigma^k$ in place of $\Sigma$.
	
	\item Evaluate $r = (c_1- f)^2 - (c_2 - f)^2$.

	\item Update $u^{k+1}$ by \Cref{eq:sbuij,eq:sbuij2} using one sweep of Gauss-Seidel.
	
	\item Update $d^{k+1} = ((d^x)^{k+1}, (d^y)^{k+1})$ by \Cref{eq:sbdij}.
	
	\item Update $b^{k+1} = ((b^x)^{k+1}, (b^y)^{k+1})$ by \Cref{eq:sbbij}
	
	\item Repeat from step 1 if $\norm{u^{k+1} - u^k} > \textrm{tol}$.
\end{enumerate}
With regards to step 4, the authors in \cite{goldstein2010geometric} recommend just one sweep of Gauss-Seidel rather than iterating to full convergence. The reason being that any advantage would be lost against the error in the updates to the Bregman parameter, $b^{k+1}$.



\chapter{Conclusion}
From the beginning to the end, we sought to provide an overview of image segmentation, building from a mathematical modelling perspective which then gave way to numerics, discretization, implementation and practical performance. 

Starting from the review of two influential image segmentation models, GAC and ACWE, we examined key modelling principles which were then unified into a convex segmentation model. We also investigated the use of two numerical methods to generate solutions with the first involving the Euler-Lagrange equations and the second the Split Bregman method. Driving the Euler-Lagrange equations to steady state proved to be a powerful technique that could be applied to the three models considered. On the other hand the Split Bregman, a convex optimization algorithm for $\ell_1$-minimization problems, outperformed the previous method by a wide margin where it was applicable, namely with the GCS model. That, however, was not the end of the story as our examples showed there is a range of parameters where the Split Bregman is blazingly fast, but there is also a range for which the Split Bregman performs sluggishly.

Looking forward, we have already mentioned the need to accelerate the solution procedure when $\lambda$ is small. Would a different optimization algorithm be suitable or is there some way to incorporate the same/similar idea that was used to accelerate the evolution of the GAC model? It was also shown that a similar convexification procedure could be applied to the piecewise smooth Mumford-Shah model. In \cite{bresson2007fast}, the authors proposed solution via a dual gradient projection method, but exploring the use of the Split Bregman there may be worthwhile.


%   BACK MATTER  %%%%%%%%%%%%%%%%%%%%%%%%%%%%%%%%%%%%%%%%%%%%%%%%%%%%%%%%%%%%%%
%
%   References and appendices. Appendices come after the bibliography and
%   should be in the order that they are referred to in the text.
%
%   If you include figures, etc. in an appendix, be sure to use
%
%       \caption[]{...}
%
%   to make sure they are not listed in the List of Figures.
%

\backmatter%
	\addtoToC{Bibliography}
	\bibliographystyle{plain}
	\bibliography{../../master_ref} %% compile bibtex>pdflatex

\begin{appendices} % optional
	\chapter{The Split Bregman method}
	\label{appdx:sb}
	%\chapter{Code}
\end{appendices}
\end{document}
