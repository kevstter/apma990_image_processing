\chapter{Geodesic Active Contours}
\label{ch:gac}
The first segmentation model we review is the geodesic active contours (GAC) model from Caselles, Kimmel, and Sapiro \cite{caselles1997geodesic}. Their model of segmentation is the (near) automaticatic selection of an object or objects from an image once an initial contour $\C_0$ is prescribed. To get an active contour $\C = \C(t)$ with $\C(0) = \C_0$, they argued for the minimization of the energy 
\begin{align}
E_\mathrm{GAC}(\C) = \int^{L(\C)}_0 g\left( \abs{\nabla f } \right) \, ds,
\label{eq:egac}
\end{align}
where 
\begin{align*}
f&: \textrm{initial image},
\\
L(\C) &: \textrm{the length of the contour $\C$,} 
\\ 
g(\xi)&: \textrm{an edge indicator function}.
\end{align*}
The energy $E_\textrm{GAC}$ may be interpreted as a weighted arc length. If $g(\xi) = 1$, we would recognize $\int_\C  ds$ as standard Euclidean arc length. Hence by designing $g$ is be small near edges and large over homogeneous regions, one would expect the contour $\C$ to be drawn to and remain at object edges when minimizing $E_\textrm{GAC}$. As an example, one may set $g$ as 
\begin{align*}
g_1(\xi) 
= \frac{1}{1 + \alpha\xi^2},
\quad\text{or}\quad 
g_2(\xi) 
= \exp(-\beta\xi^2).
\end{align*}
Notice that as $\abs{\xi} \rightarrow \infty$, $g_1,g_2 \rightarrow 0$.  

In the next section, we will derive the Euler-Lagrange equation and minimize $\egac$ by gradient descent. However, before we move on, let us note that we are seeking local minima when minimizing $\egac$ as this energy does have the unfortunate feature that it attains a global minimum of zero when the contour contracts to a point and vanishes. We will revisit this issue when reviewing the results of the numerical examples.


\section{Euler-Lagrange equation to the GAC model}
The derivation will be helped tremendously with some setup and a quick review of vector calculus. Let $\C: [0,1]\rightarrow \mathbf{R}^2$ be a parametrized curve, $\C = \C(p)$. We then can relate the arc length element to the parameter $p$ as $ds = \abs{C'(p)} dp$. We also want to recognize the unit tangent vector, $\tang$, unit (inward) normal, $\normal$, and curvature, $\kappa$, as 
\begin{align*}
\tang 
= \frac{\C'}{\abs{\C'}},\quad 
\normal 
= \frac{ \tang '}{\abs{\tang'}},
\quad\text{and}\quad 
\kappa = \frac{ \abs{\tang'} }{ \abs{ \C' } },
\end{align*}
with the prime notation denoting differentiation w.r.t. $p$. We will also assume $\C(0) = \C(1)$. 

Next, rewriting $\egac$ as 
\begin{align*}
\egac = \int^{L(\C)}_0 g \, ds 
= \int^1_0 g(\C) \abs{\C'(p)} \, dp ,
\end{align*} 
we can compute its first variation: 
\begin{align*}
\dd{}{\gamma} \int^1_0 g(\C + \gamma h) \abs{\C' + \gamma h'} \, dp \bigg\rvert_{\gamma = 0}
&=  \int^1_0 \nabla g(\C) \cdot h \abs{\C'} + g(\C) \frac{\C'}{\abs{\C'}} \cdot h' \,dp 
\\
&=\int^1_0 \nabla g(\C) \cdot h \abs{\C'} + g(\C) \tang \cdot h' \,dp 
\\
&=\int^1_0 \nabla g \cdot h \abs{\C'} - ( g \tang)' \cdot h \, dp
\\
&=\int^1_0 \nabla g \cdot h \abs{\C'} - (\nabla g \cdot \C')(\tang \cdot h) - g\tang ' \cdot h \, dp
\\
&=\int^1_0 \nabla g \cdot h \abs{\C'} - (\nabla g \cdot \tang \abs{\C'} )(\tang \cdot h) - g\abs{\tang'} \normal \cdot h \, dp
\\
&=\int^1_0 \big[ \nabla g  - (\nabla g \cdot \tang )\tang \big] \cdot h \abs{\C'} - g\kappa \abs{\C'} \normal \cdot h \, dp
\\
&=\int^1_0 \big[\nabla g \cdot \normal \normal \big] \cdot h \abs{\C'} - g\kappa  \normal \cdot h \abs{\C'} \, dp
\\
&=\int^1_0 \big[ \nabla g \cdot \normal - g\kappa \big] \normal  \cdot h \abs{\C'} \, dp.
\\ 
\end{align*}
This gives a gradient descent evolution equation $\C_t = \left( g(\C) \kappa - \nabla g(\C) \cdot \normal \right) \normal $.
For the level set formulation, with $\phi = \phi(x, t)$ as the level set function, we have 
\begin{align}
\begin{split} 
\phi_t 
&= \left(
g( \abs{\nabla I } ) \Div\left( \frac{\nabla \phi}{\abs{\nabla \phi}} \right)
	+  \nabla g(  \abs{\nabla I } )\cdot  \frac{\nabla \phi}{\abs{\nabla \phi}}
\right) \abs{ \nabla \phi }
\\
&= \abs{\nabla \phi} \Div\left( 
g(  \abs{\nabla I } ) \frac{\nabla \phi}{\abs{\nabla \phi}}
\right) .
\end{split}
\label{eq:gac_ls}
\end{align}

\section{Numerical discretization and examples}
To discretize \eqref{eq:gac_ls}, we will use the semi-implicit Gauss-Seidel type numerical scheme proposed by Aubert and Vese \cite{aubert1997variational} and used to good effect by Chan and Vese with their ACWE model \cite{chan2001active} rather than using explicit forward Euler with second order centred differences as originally suggested\footnote{Although to be fair, it was suggested for ease of implementation as a proof of concept. For us, the same semi-implicit numerical scheme can and will be used again in \Cref{ch:acwe,ch:gcs} so it makes sense to give a complete presentation once and be able to reuse essentially the same code three times. } in \cite{caselles1997geodesic}.

\section{in some later section} 

slightly asymmetry

unable to capture interior contours automatically