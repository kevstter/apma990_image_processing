\chapter{Introduction}
The aim of this project is to present to the reader several core techniques relevant to image segmentation. At the top level is modelling the image segmentation problem and producing a relevant mathematical expression. Each of the models we review is designed around principles based on image characteristics, whether it is edge-based, where we have the geodesic active contour (GAC) model, or region-based with the active contours without edges (ACWE) model. The mathematical model, or ``energy'', is then minimized to give the desired effect, for instance producing a contour that wraps along the edges of an object in a photo, or assigns membership of pixels to divide an image into distinct regions. 

When a model is set, the task then turns to seeking and computing a minimizer. A standard technique is minimization via gradient descent of the Euler-Lagrange equation. This PDE-based technique is extremely general and can be applied with great success to each of the segmentation models we consider. However when the model is convex, powerful algorithms from convex optimization may be more efficient. Along this direction, we will present a unified globally convex segmentation (GCS) model which is an amalgam of the GAC and ACWE model, and its solution by the Split Bregman method, an extremely efficient convex optimization algorithm for $\ell_1$-minimization problems.

Finally, as is the case in any image processing tasks, we have to ask if the practical performance matches the theoretical predictions. To this end, numerical examples will be shown throughout, but with the majority being in \Cref{ch:gcs} where we have the unified model and two minimization algorithms matched up against each other.


Lastly, we also point out that there is but one more section in this introduction and it is a list of standardized notation used starting in \Cref{ch:gac} and all through to the end. So print it out, laminate it, open an extra copy on all your devices, send it to all your friends and family, and then move on to the next chapter. 


\begin{itemize}
	\item Proceed as follows: present 2 models: GAC and ACWE, derive their E-L and show its capabilities and limitations.
	
	\item Convexification of ACWE and a combined model: GCS, derive E-L with regularization. Give split bregman solution
	
	\item Conclusion
\end{itemize}

\section{Notation}
\begin{itemize}
	\item $f$: the initial/given 2-D image.
	
	\item $\Omega$: image domain.
	
	\item $\Sigma, \Sigma_1,$ etc.: subsets of the image domain $\Omega$.
	
	\item $\C$: curve/contour, typically closed with well-defined inside and outside. This may also represent a collection of (closed) contours, $\C = \cup_{k \in \mathcal{I}} \C_k$.
	
	\item $\phi$: level set function whose zero level set represents contour $\C$ at time $t$, i.e. $\C(t) = \{(x,y) \in \Omega \mid \phi(x,y,t) = 0\}$. 
	
	\item $t$: time parameter. 
	
	\item $s$: arc length parameter.
	
	\item $H$: Heaviside function
	
	\item $\kappa$: curvature. 
	
	\item $\lambda, \mu, \theta, \varepsilon, \eta$: regularization parameters.
	
	\item $\mathbf{1}_\Sigma$: indicator/characteristic function of the subset $\Sigma$, i.e. $\mathbf{1}_\Sigma(x) = 1 \iff x \in \Sigma$.
	
\end{itemize}