\chapter{The Split Bregman method}
\label{appdx:sb}
Let $E, H: \mathbf{R}^n \rightarrow \mathbf{R}$ be two convex functionals with $\min_u H(u) = 0$ and suppose $H$ is differentiable. The Bregman iteration proposes solving the unconstrained optimization problem 
\begin{align*}
\min_u E(u) + \lambda H(u)
\end{align*}
by the 2-step iterative process 
\begin{align}
\begin{split} 
u^{k+1} &= \argmin_u \left\{ E(u) - \ip{p^k}{u - u^k} + \lambda H(u)  \right\}
\\
p^{k+1} &= p^k - \lambda\nabla H(u^{k+1}).  
\end{split}
\label{eq:breg_its}
\end{align}
Detailed discussion for the above iterative process can be found in \cite{osher2005iterative}. We note three key findings and a fourth one from \cite{goldstein2010geometric}. First is the following convergence result.
\begin{thm}
	Assume the solutions to the subproblems of \eqref{eq:breg_its} exists. If $u^*$ is a minimizer of $H$ and $E(u^*) < \infty$, then we have 
	\begin{enumerate}
		\item $H(u^{k+1}) \leq H(u^{k})$, and 
		\item $H(u^k) \leq H(u^*)  + E(u^*) / k$.
	\end{enumerate}
\end{thm} 
The next result concerns the special case (and for us, the relevant case) where $H$ is of the form $\frac{1}{2}\norm{Au - b}^2_2$, where $A$ is linear. 

\begin{thm}
	With the additional requirements on $H$, we have the following:
	\begin{enumerate}
		\item The iterative process \eqref{eq:breg_its} is well-defined and existence of solutions to each subproblem is guaranteed.
		
		\item The iterative process \eqref{eq:breg_its} is equivalent to
		\begin{align}
		\begin{split} 
		u^{k+1} &= \argmin_u \left\{ 
		E(u) + \frac{\lambda}{2} \norm{A u - b^k}^2_2
		\right\},
		\\
		b^{k+1} &= b^k + b - Au^{k+1}.
		\end{split}
		\label{eq:breg_its2}
		\end{align}
		
		\item If some iterate, $u^*$, satisfies $Au^* = b$, then $u^*$ is a solution to the constrained optimization problem 
		\begin{align}
		\min_u E(u) \quad\text{such that}\quad Au = b.
		\label{eq:constrained}
		\end{align}
	\end{enumerate}
\end{thm}
Combined, the theorems validate \eqref{eq:breg_its2} as an iterative solution process for \eqref{eq:constrained}.

The Split Bregman is a further specialization for $L_1$-regularized problems. Suppose we seek the solution to $\min_u E(u)$ and that $E$ may written as $E(u) = \norm{\Phi(u)}_1 + H(u)$. The Split Bregman proposes decoupling the problem by first introducing an auxiliary variable, $d$, and expressing as the constrained minimization problem 
\begin{align*}
\min_{u,d} \norm{ d}_1 + H(u)
\quad\text{such that}\quad d = \Phi(u).
\end{align*}
Identifying $A(u,d) = d - \Phi(u)$ and applying the Bregman iteration \eqref{eq:breg_its2} we have 
\begin{align}
(u^{k+1}, d^{k+1}) &= \argmin_{u,d} \left\{
\norm{d}_1 + H(u) + \frac{\lambda}{2}\norm{d - \Phi(u) - b^k}^2_2 
\right\}
\label{eq:first_step}
\\ 
b^{k+1} 
&= b^k - (d^{k+1} - \Phi(u^{k+1})).
\end{align}
The Split Bregman ``splits'' the first step \eqref{eq:first_step} into 2 substeps
\begin{align}
u^{k+1} &= \argmin_u H(u) + \frac{\lambda}{2} \norm{d^k - \Phi(u) - b^k}^2_2 ,
\label{eq:spleet}
\\
d^{k+1} &= \argmin_d \norm{d}_1 + \frac{\lambda}{2} \norm{d - \Phi(u^{k+1}) - b^k}^2_2, 
\label{eq:spleet2}
\end{align}
and iterates to convergence. The efficiency of the Split Bregman is reliant on whether one has fast solutions to the substeps \eqref{eq:spleet} and \eqref{eq:spleet2}. \Cref{subsect:sb} provides a detailed example.


