\chapter{Globally Convex Segmentation}
\label{ch:gcs}
Our numerical examples with the two models, GAC and ACWE, show still plenty of room for improvement from the modelling perspective. Of particular concern is this knack for finding and getting stuck at local minima (by design, in case of the GAC model), and therefore the solution we get is dependent on the initial contour we set. These issues are address by Chan, Esedo\={g}lu, and Nikolova \cite{chan2006algorithms}, and their ideas further refined by Bresson et al. \cite{bresson2007fast}, in what they referred to as the convexification of the ACWE model and the unification of GAC and ACWE models.

In this chapter our viewpoint on the segmentation problem shifts somewhat. With the two previous models, the expression and evolution of the active contour, in the form of a level set function, was at the forefront. When we turn to the globally convex segmentation (GCS) model, the viewpoint shifts to subsets and indicator functions of these subsets. Keep in mind that for 2-phase segmentation any subset $\Sigma \subset \Omega$, and equivalently the corresponding indicator function $\ind_\Sigma$, defines a segmentation of the image domain. Thus the objective is to determine an indicator function $u(x) = \ind_\Sigma(x)$, and a piecewise constant solution $\widetilde u(x) = c_1 \ind_\Sigma(x) + c_2 (1 - \ind_\Sigma(x))$. 

First we address the non-convexity in the ACWE model. Restating the optimization problem in terms of indicator functions, we have 
\begin{align}
\min_{\substack{\Sigma\subset\Omega \\ 
		u(x) = \ind_\Sigma(x)}} 
\left\{\eacwe(u, c_1, c_2; \lambda)
= \int_{\Omega} \abs{\nabla u} \dx 
+ \int_\Omega u(c_1 - f)^2   + (1-u)(c_2 - f)^2 \dx 
\right\}.
\label{eq:eacwe2}
\end{align}
Observe that the function set we are optimizing over is not convex. For instance suppose $\Sigma_1, \Sigma_2 \subset \Omega$, $\Sigma_1 \cap \Sigma_2 = \emptyset$, $\Omega \setminus (\Sigma_1 \cup \Sigma_2) \neq \emptyset$ and set $u_1 = \ind_{\Sigma_1}$ and $u_2 = \ind_{\Sigma_2}$. Then any convex combination $w = ku_1 + (1-k)u_2$, $k \in (0,1)$, would be a function that takes on three values. Addressing the issue of convexity are two key theorems presented in the next section which relaxes the restriction on our function space, allows $u$ to take on the continuum of values between 0 and 1, and leads to an equivalent but convex minimization problem.

\section{A unified GAC \& ACWE convex segmentation model}
The key observation made by Chan et al. \cite{chan2006algorithms} is that in choosing a non-compactly support $H_\eta$ for the gradient descent evolution \eqref{eq:acwe_el}, the following gradient flow will have the same steady state solutions: 
\begin{align*}
\phi_t = \Div\left(\frac{\nabla \phi}{\abs{\nabla \phi}} \right) 
- \lambda (c_1 - f)^2 + \lambda (c_2 - f)^2 .
\end{align*}
This in turn is the gradient descent equation the following energy:
\begin{align}
E(\phi, c_1, c_2) 
= \int_\Omega \abs{\nabla \phi} \dx 
+ \lambda \int_\Omega \big[ (c_1 - f)^2 - (c_2 - f)^2 \big] \phi \dx ,
\label{eq:mod_acwe}
\end{align}
which gets us to the first of two key theorems from \cite{chan2006algorithms}.

\begin{thm}
	For any given fixed $c_1, c_2 \in \mathbf{R}$, a global minimizer for $\eacwe(\cdot, c_1, c_2; \lambda)$ can be found by carrying out the following convex minimization: 
	\begin{align}
	\min_{0\leq u \leq 1}\left\{
		\widetilde E_{\textrm{ACWE}} (u, c_1, c_2; \lambda)
		=
		\int_{\Omega} \abs{\nabla u} \dx 
	+ \lambda\int_{\Omega} \big[ (c_1 - f(x) )^2 - (c_2 - f(x))^2 \big] u(x) \dx
	\right\}
	\label{eq:thm1}
	\end{align}
	and then setting $\Sigma = \{ x \mid u(x) \geq \zeta \}$ for a.e. $\zeta \in [0, 1]$.
	\label{thm:1}
\end{thm}
The theorem above links the two energies, $\eacwe$ and $\widetilde E_\textrm{ACWE}$,
and guarantees that any global minimum of \eqref{eq:thm1} is only a thresholding step away from a global minimum of the ACWE model. The next theorem is but one way to tackle the minimization problem \eqref{eq:thm1}.

\begin{thm}
	Let $r(x) \in L^\infty(\Omega)$. Then the convex, constrained minimization problem
	\begin{align*}
	\min_{0 \leq u \leq 1} \int_{\Omega} \abs{\nabla u} \dx + \lambda \int_{\Omega} r(x) u \dx 
	\end{align*}
	has the same set of minimizers as the following convex, unconstrained minimization problem:
	\begin{align*}
	\min_u \int_{\Omega} \abs{\nabla u} \dx + \lambda \int_{\Omega} r(x) u + \alpha \nu(u) \dx 
	\end{align*}
	where $\nu(z) = \max\{ 0 , 2\abs{z - \frac{1}{2} } - 1\}$, provided that $\alpha > \frac{\lambda}{2}\norm{r(x)}_{L^\infty(\Omega)}$.
	\label{thm:2}
\end{thm}
The addition of the exact penalty term \cite{hiriart1993convexI,hiriart1993convexII} grants us an unconstrained formulation. Deriving the Euler-Lagrange equation is straightforward in this form. But before doing so, we add one more modification to give a unified globally convex segmentation (GCS) model.

Per Bresson et al. \cite{bresson2007fast}, building on the work of \cite{chan2006algorithms}, they propose the minimization 
\begin{align}
\min_{0\leq u \leq 1} \left\{
\egcs(u, c_1, c_2; \lambda) 
= 
\int_{\Omega} g(x)  \abs{\nabla u} \dx 
+ \lambda\int_{\Omega} \big[ (c_1 - f(x) )^2 - (c_2 - f(x))^2 \big] u(x) \dx
\right\},
\label{eq:min_gcs}
\end{align}
Note the similarity to \eqref{eq:thm1}. The only change is the added edge indicator function $g(x)$ in the first term. Note also the close relationship to $\egac$ from \Cref{ch:gac}. If $u$ is an indicator function, then 
\begin{align*}
\egcs(\ind_\Sigma(x), c_1, c_2; \lambda) 
&= \int_{\Omega} g(x)  \abs{\nabla \ind_\Sigma(x)} \dx 
+ \lambda\int_{\Omega} \big[ (c_1 - f(x) )^2 - (c_2 - f(x))^2 \big] \ind_\Sigma(x) \dx
\\
&= \egac(\partial \Sigma) +  \lambda\int_{\Omega} \big[ (c_1 - f(x) )^2 - (c_2 - f(x))^2 \big] \ind_\Sigma(x) \dx,
\end{align*}
reduces to the familiar GAC energy but now subject to an ACWE-type fitting energy constraint. Finally, we point out that $\egcs$ satisfies statements same as those in Theorem \ref{thm:1} and \ref{thm:2}, but with $\int g\abs{\nabla u}$ in place of $\int \abs{\nabla u}$.

The next section will not be just one, but two numerical algorithms for image segmentation under the GCS model. In addition to the usual Euler-Lagrange equation and gradient descent, we will also discuss application of the Split Bregman method \cite{goldstein2010geometric,goldstein2009split} to this segmentation model.

\section{Minimization of the GCS model: gradient descent and the Split Bregman}
For the following sections, let 
\begin{align*}
r(x, c_1, c_2) = (c_1 - f)^2 - (c_2 - f)^2.
\end{align*}

\subsection{Euler-Lagrange and numerical discretization for the GCS model}
For this subsection, we are considering the unconstrained minimization problem 
\begin{align*}
\min_u \underbrace{\int_{\Omega } g\abs{\nabla u} \dx 
+ \int_{\Omega} \lambda r u + \alpha \nu(u) \dx }_{=\widetilde E_{\textrm{GCS}}(u)},
\end{align*}
($\egcs$ with an exact penalty term, see Theorem \ref{thm:2}).
Fixing $c_1, c_2$ and $\lambda$, we have
\begin{align*}
\dd{}{\gamma} \widetilde E_\textrm{GCS}(u + \gamma h)
\bigg\rvert_{\gamma = 0} 
&= \int_{\Omega} g \frac{\nabla u}{\abs{\nabla u}} \cdot \nabla h
+ \left( \lambda r + \alpha \nu'(u) \right) h \dx 
\\
&= \int_{\Gamma} h g \frac{\nabla u}{\abs{\nabla u}} \cdot \normal  \, d \Gamma - \int_{\Omega } \Div\left( g \frac{\nabla u}{\abs{\nabla u}} \right) - \lambda r - \alpha \nu'(u) \dx ,
\end{align*}
which gives us the descent equation 
\begin{align}
\label{eq:gcs_el}
u_t = \Div\left( g(x)  \frac{\nabla u}{\abs{\nabla u}} \right) - \lambda r(x,c_1,c_2)- \alpha \nu'(u)
\end{align}
and boundary condition $g\frac{\nabla u}{\abs{\nabla u}} \cdot \normal = 0$. 

Those familiar with exact penalty functions will note a small techical issue with the descent equation. As defined in Theorem \ref{thm:2}, $\nu(z)$ has two points of non-differentiability at $z = 0$ and $z = 1$. It will be necessary to substitute with a smooth penalty function, $\nu_\mu(z)$. In our numerical examples, we use
\begin{align*}
\nu_\mu(z) = \begin{cases}
-2z - \mu/2, & z < -\frac{\mu }{2}
\\
\frac{2}{\mu} z^2, &-\frac{\mu}{2} \leq z < 0
\\
0, & 0\leq z \leq 1
\\
\frac{2}{\mu}(z-1)^2, & 1 < x \leq 1 + \frac{\mu }{2}
\\
2(z - 1) - \frac{\mu }{2}, & z > 1+\frac{\mu }{2}.
\end{cases}
\end{align*}
This choice is similar to one analyzed by Pinar and Zenios \cite{pinar1994smoothing}.

The numerical discretization of \eqref{eq:gcs_el} is almost a carbon copy of our work in \Cref{sec:gac_num_disc,sec:3.2}. The Gauss-Seidel iterations are (if working through $(i,j)$ in the usual order)
\begin{align*}
u^{n+1}_{ij} 
= \frac{1}{a_0} 
\left( u^n_{ij} + \Delta t 
\left( a_1 u^n_{i+1,j} + a_2 u^{n+1}_{i-1,j} + a_3 u^{n}_{i,j+1} + a_4 u^{n+1}_{i,j-1}
- \lambda r_{ij}^n
- \alpha \nu'_\mu( u^n_{ij} )
\right)
\right),
\end{align*}
with $a_1,a_2,a_3,a_4$ as given in \Cref{sec:gac_num_disc} and $a_0 = 1 + \Delta t(a_1 + a_2 + a_3 + a_4)$. Perhaps more important to discuss is the role of the thresholding step from Theorem \ref{thm:1}. Recall that we are using the alternating minimization scheme to first set $c_1$ and $c_2$, then minimizing $\egac(\cdot, c_1, c_2)$ to determine $u$. In practice, this means we set $\Sigma = \{x \mid u(x) \geq 0.5\}$ whenever it is necessary to update $c_1$ and $c_2$, and when presenting the steady state solution and the contour $\C$.


\subsection{The Split Bregman method applied to the GCS model}
From the beginning we have said that minimization by way of the Euler-Lagrange equation is an extremely general technique and have now seen it applied to two non-convex and one convex segmentation model. With a convex model, it would be negligent to not consider efficient solvers from convex optimization. 

In \cite{goldstein2010geometric}, Goldstein, Bresson and Osher applied the Split Bregman method \cite{goldstein2009split} to the GCS image segmentation model with good results. One advantage of using the Split Bregman method is that we are able to solve the minimization problem \eqref{eq:min_gcs} without the need of the exact penalty function  or any added regularization required due to division by $\abs{\nabla u}$. We will compare and contrast this alternative to the earlier gradient descent approach. Additional details of the Bregman and Split Bregman are provided in the Appendix \ref{appdx:sb}.  

To derive the Split Bregman formulation, we start with expressing \eqref{eq:min_gcs} in discrete form: 
\begin{align*}
\min_{0\leq u \leq 1} \norm{ \nabla u }_g + \lambda \ip{r}{u},
\end{align*}
where $\norm{ \nabla u }_g = \norm{ g \nabla u}_1$. Next, 
define an auxiliary variable $d = (d^x, d^y) = (u_x, u_y) = \nabla u$ and introduce a quadratic penalty function to weakly enforce this equality to get
\begin{align*}
\min_{0\leq u \leq 1, d} 
 \norm{d }_g + \lambda  \langle r, u \rangle
+ \frac{\theta}{2} \norm{ d - \nabla u }_2^2.
\end{align*} 
The Bregman iteration, for this particular problem, is the 2-step procedure (see \cite[Theorem 2.2.]{goldstein2009split})
\begin{align}
(u^{k+1}, d^{k+1}) 
&= \argmin_{0\leq u \leq 1, d}  \norm{d}_g + \lambda\ip{r}{u} + \frac{\theta}{2} \norm{d - \nabla u - b^k }^2_2 
\\
b^{k+1} &= b^k - (d^{k+1} - \nabla u^{k+1}) .
\label{eq:sbbij}
\end{align} 
The Split Bregman then ``splits'' the first step into 2 substeps that may be iterated until convergence, effectively decoupling $u$ and $d$: 
\begin{align}
u^{k+1} &= \argmin_{0\leq u \leq 1} \lambda \ip{r}{u}
+ \frac{\theta}{2} \norm{d^k - \nabla u - b^k }^2_2,
\label{eq:splitu}
\\
d^{k+1} 
&= \argmin_d \norm{d}_g + \frac{\theta}{2} \norm{d - \nabla u^{k+1} - b^k}^2_2.
\label{eq:splitd}
\end{align}
Each of the substeps can be solved efficiently. We begin by tackling the solution to \eqref{eq:splitd}.

To begin, let us specify precisely the meaning of the vector norm $\norm{\cdot }_1$. The variant we give here is sometimes referred to as the isotropic TV-norm,
\begin{align*}
\norm{ d }_1
= \sum^N_{i,j = 1}  \norm{d_{ij}}_2 
= \sum^{N}_{i,j = 1} \sqrt{ (d^x_{ij} )^2 + (d^y_{ij} )^2 },
\end{align*}
with which we can express \eqref{eq:splitd} as
\begin{align*}
\sum^{N}_{i,j = 1} g_{ij} \sqrt{ (d^x_{ij} )^2 + (d^y_{ij} )^2 }
+ \frac{\theta}{2} \sum^N_{i,j=1} \big[ (d^x_{ij} - \nabla_x u^{k+1}_{ij} - (b^x_{ij})^k)^2 
+ (d^y_{ij} - \nabla_y u^{k+1}_{ij} - (b^y_{ij})^k)^2 
\big]
\eqqcolon \sum^N_{i,j=1} S_{ij}
\end{align*}
For each $(i,j)$ we can optimize by elementary calculus, 
\begin{align*}
0
=\pd{S_{i,j}}{d^x_{i,j}} 
= g_{ij}\frac{d^x_{ij}}{ \norm{d_{ij}}_2}
+ \theta (d^x_{ij} - \nabla_x u^{k+1}_{ij} - (b^x_{ij})^k) 
\iff
\nabla_x u^{k+1}_{ij} + (b^x_{ij})^k  
= d^x_{ij} \left( 1 + \frac{g_{ij}}{\theta \norm{d_{ij}}_2}
\right)
\\
0
=\pd{S_{ij}}{d^y_{i,j}} 
= g_{ij}\frac{d^y_{ij}}{ \norm{d_{ij}}_2}
+ \theta (d^y_{ij} - \nabla_y u^{k+1}_{ij} - (b^y_{ij})^k) 
\iff
\nabla_y u^{k+1}_{ij} + (b^y_{ij})^k 
= d^y_{ij} \left( 1 + \frac{g_{ij}}{\theta\norm{d_{ij}}_2}
\right)
\end{align*}
which combines as 
\begin{align*}
\norm{\nabla u^{k+1}_{ij} + b^k_{ij}}_2^2 = \left(\norm{ d_{ij}}_2 + \frac{g_{ij}}{\theta}
\right)^2
\implies 
\norm{d_{ij}}_2 = \norm{\nabla u^{k+1}_{ij} + b^k_{ij} }_2 - \frac{g_{ij}}{\theta},
\end{align*}
(if a solution exists), so
\begin{align*}
d_{ij}^x 
= \frac{ \nabla_x u^{k+1}_{ij} + (b^x_{ij})^k }{ \norm{\nabla u^{k+1}_{ij} + b^k_{ij} }_2} \left( \norm{\nabla u^{k+1}_{ij} + b^k_{ij} }_2 - \frac{g_{ij}}{\theta} \right), 
\end{align*}
and similarly for $d^y_{ij}$. Otherwise, there is one other critical point giving the minimum at $d = 0$. The full solution may be presented as
\begin{align}
d^{k+1}
= \max\left( \norm{ \nabla u^{k+1} + b^k}_2 - \frac{g}{\theta}, 0 \right) \frac{ \nabla u^{k+1} + b^k}{ \norm{ \nabla u^{k+1} + b^k}_2}.
\label{eq:sbdij}
\end{align}

The solution to \eqref{eq:splitu} can be found by first differentiating to get $0 = \lambda r  +\theta  \nabla \cdot ( d^k - \nabla u - b^k) \implies \Delta u_{ij} = \frac{\lambda}{\theta} r_{ij} + \nabla \cdot (d^k_{ij} - b^k_{ij})$. As the energy \eqref{eq:splitu} is quadratic in $u_{ij}$, either the optimal value is this unconstrained minimizer, call it $\widetilde u_{ij}$, or if $\widetilde{u}_{ij} \notin [0,1]$, the optimal value we seek is $0$ or $1$, whichever is nearer. For instance with centred differencing of the Laplacian, we have 
\begin{align}
\widetilde u_{ij} = \frac{1}{4}\left(
u_{i+1,j} + u_{i-1,j} + u_{i,j+1} + u_{i,j-1} 
- \frac{\lambda}{\theta} r_{ij}
-  (D^-_x, D^-_y) \cdot  (d^k_{ij} - b^k_{ij} )
\right) 
\label{eq:sbuij}
\end{align}
and 
\begin{align}
u_{ij}^{k+1}  = \max \{ \min\{ \widetilde u_{ij}, 1 \} , 0 \}.
\label{eq:sbuij2}
\end{align}
The full algorithm, to summarize, is
\begin{enumerate}
	\item Threshold $u^k$ to determine $\Sigma^k$, eg. $\Sigma^k = \{x\in\Omega \mid u(x) \geq 0.5 \}$.
	\item Update $c_1, c_2$ by \Cref{eq:c1,eq:c2} with $\Sigma^k$ in place of $\Sigma$.
	
	\item Evaluate $r = (c_1- f)^2 - (c_2 - f)^2$.

	\item Update $u^{k+1}$ by \Cref{eq:sbuij,eq:sbuij2} using one sweep of Gauss-Seidel.
	
	\item Update $d^{k+1} = ((d^x)^{k+1}, (d^y)^{k+1})$ by \Cref{eq:sbdij}.
	
	\item Update $b^{k+1} = ((b^x)^{k+1}, (b^y)^{k+1})$ by \Cref{eq:sbbij}
	
	\item Repeat from step 1 if stopping criteria not met.
\end{enumerate}
With regards to step 4, the authors in \cite{goldstein2010geometric} recommend just one sweep of Gauss-Seidel rather than iterating to full convergence, arguing that any advantage from iterating to full convergence would be lost to the error in the updates to the Bregman parameter, $b^{k+1}$.


\section{Examples and discussion}
In the examples to follow, the GCS model \eqref{eq:min_gcs} is solved by two numerical algorithms. The parameters used consistently throughout are summarized in \Cref{tab:model parameters};  $\lambda$ will be provided with each example. The same edge indicator function \eqref{eq:edge_indicator} from \Cref{ch:gac} is used and initialize $u$ with the given image. A discrete energy is calculated at each step and the stopping criteria is met when the relative change of this discrete energy drops below $9\times 10^{-6}$. We will be looking for any differences in the quality of the segmentation, and comparing efficiency of each method by their iteration count.

\begin{table}[htb!]
	\caption[]{Discretization and model parameters.}
	\centering
	\begin{tabular}{lll c ll} \toprule[1.25pt]
		\multicolumn{3}{c}{Euler-Lagrange (E-L)} & \phantom{abc} & \multicolumn{2}{c}{Split Bregman (SB)}
		\\ \midrule
		$h = 1$ & $\Delta t = 5\times 10^{-4}$ & $\alpha = \lambda$ & & $h = 1$ & $\theta = 0.5$
		\\
		$\varepsilon=10^{-6}$ & $\mu = 10^{-9}$
		\\ \bottomrule[1.25pt]
	\end{tabular}
	\label{tab:model parameters}
\end{table}

We start by revisiting the brain scan last seen in \Cref{ch:acwe}. \Cref{fig:brain_gcs} shows the segmentation results by both methods. We find here that the Euler-Lagrange (E-L) solution gives (at the same $\lambda$) slightly smoother results: smoother contours and larger connected regions than smaller disconnected ones when compared to the Split Bregman (SB) solution. The SB method is much more efficient, needing only 26 iterations compared to 626 for E-L method. This phenomenon is clearly illustrated in \Cref{fig:sb_energy} where the energy decrease happens almost entirely within the first 5 iterations for the SB method, but much more gradually (over 600 iterations) for E-L method.
\begin{figure}[htb!]
	\centering
	\begin{minipage}{0.31\textwidth}
		\includegraphics[width=\textwidth]{brain0}
	\end{minipage}%
	\begin{minipage}{0.31\textwidth}
		\includegraphics[width=\textwidth]{braingcs0}
	\end{minipage}%
	\begin{minipage}{0.31\textwidth}
		\includegraphics[width=\textwidth]{braingcssb}
	\end{minipage}
	\caption{(Left) Initial image. (Middle) Euler-Lagrange solution. (Right) Split Bregman solution.}
	\label{fig:brain_gcs}
\end{figure}

\begin{figure}[htb!]
	\centering
	\begin{minipage}{0.40\textwidth}
		\includegraphics[width=\textwidth]{el_E}
	\end{minipage}\quad
	\begin{minipage}{0.40\textwidth}
		\includegraphics[width=\textwidth]{sb_E}
	\end{minipage}%
	\caption{Discrete energy when evolving the GCS model by (left) Euler-Lagrange iterations, and (right) Split Bregman iterations.}
	\label{fig:sb_energy}
\end{figure}

Next we investigate whether the performance of the SB method differs over other choices of $\lambda$. We do this with 2 images: the same brain scan image from \Cref{fig:brain_gcs}, and the \textsc{Matlab} favorite, ``cameraman.tif'', shown in \Cref{fig:cam}.

\begin{figure}[htb!]
	\centering
	\begin{minipage}{0.40\textwidth}
		\includegraphics[width=\textwidth]{camman}
	\end{minipage}\quad
%	\begin{minipage}{0.40\textwidth}
%		\includegraphics[width=\textwidth]{sb_E}
%	\end{minipage}%
	\caption{Cameraman, a \textsc{Matlab} classic.}
	\label{fig:cam}
\end{figure}

\begin{figure}[htb!]
	\centering
	
	\begin{minipage}{0.23\textwidth}
		\includegraphics[width=\textwidth]{cammu10}
	\end{minipage}\,
	\begin{minipage}{0.23\textwidth}
		\includegraphics[width=\textwidth]{cammu10e}
	\end{minipage}\quad 
	\begin{minipage}{0.23\textwidth}
		\includegraphics[width=\textwidth]{brainmu10}
	\end{minipage}\,
	\begin{minipage}{0.23\textwidth}
		\includegraphics[width=\textwidth]{brainmu10e}
	\end{minipage}
	
	\begin{minipage}{0.23\textwidth}
		\includegraphics[width=\textwidth]{cammu1}
	\end{minipage}\,
	\begin{minipage}{0.23\textwidth}
		\includegraphics[width=\textwidth]{cammu1e}
	\end{minipage}\quad 
	\begin{minipage}{0.23\textwidth}
		\includegraphics[width=\textwidth]{brainmu1}
	\end{minipage}\,
	\begin{minipage}{0.23\textwidth}
		\includegraphics[width=\textwidth]{brainmu1e}
	\end{minipage}
	
	\begin{minipage}{0.23\textwidth}
		\includegraphics[width=\textwidth]{cammu01}
	\end{minipage}\,
	\begin{minipage}{0.23\textwidth}
		\includegraphics[width=\textwidth]{cammu01e}
	\end{minipage}\quad
	\begin{minipage}{0.23\textwidth}
		\includegraphics[width=\textwidth]{brainmu01}
	\end{minipage}\,
	\begin{minipage}{0.23\textwidth}
		\includegraphics[width=\textwidth]{brainmu01e}
	\end{minipage}
	
	\begin{minipage}{0.23\textwidth}
		\includegraphics[width=\textwidth]{cammu001}
	\end{minipage}\,
	\begin{minipage}{0.23\textwidth}
		\includegraphics[width=\textwidth]{cammu001e}
	\end{minipage}\quad 
	\begin{minipage}{0.23\textwidth}
		\includegraphics[width=\textwidth]{brainmu001}
	\end{minipage}\,
	\begin{minipage}{0.23\textwidth}
		\includegraphics[width=\textwidth]{brainmu001e}
	\end{minipage}
	\caption{(Top row) $\lambda = 10$. (Second row) $\lambda = 1$. (Third row) $\lambda = 0.1$. (Bottom row) $\lambda = 0.01$. }
	\label{fig:diff_lam}
\end{figure}

Tests with $\lambda = 10, 1, 0.1, 0.01$ are shown in \Cref{fig:diff_lam}. We find as $\lambda$ decreases, it is slower to reach steady steady and the solution begins to favor smoother contours and reject small disconnect regions. Perhaps this shouldn't be much of a surprise since as $\lambda$ decreases, the energy $\egcs$ is increasing dominated by the first term and thus the contours evolve predominately under curvature-like flow and favours shortening of contour. With that said, the SB method still outperforms the E-L method considerably in these cases.





